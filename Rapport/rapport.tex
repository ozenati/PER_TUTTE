\documentclass[12pt]{report}
%\usepackage[francais]{babel}
\usepackage{lmodern}
\usepackage[a4paper]{geometry}
\usepackage[T1]{fontenc}
\usepackage[utf8]{inputenc}  
\usepackage{moreverb}
\usepackage{amsmath}
\usepackage{amsfonts}
\usepackage{amssymb}
\usepackage{textcomp}
\usepackage{pifont}
\usepackage{geometry}
\usepackage[pdftex]{graphicx}
\usepackage{graphics}
\usepackage{url}
\usepackage{graphicx}
\usepackage{float}
\usepackage{color}
\usepackage[nottoc, notlof, notlot]{tocbibind}
\usepackage[french]{varioref}
\usepackage[Glenn]{fncychap}
\usepackage{pdfpages}

\usepackage{multicol}

\usepackage{listings}

\lstset{%configuration de listings
float=hbp,%
basicstyle=\ttfamily\small, %
columns=flexible, %
tabsize=2, %
frame=trBL, %
frameround=tttt, %
extendedchars=true, %
showspaces=false, %
showstringspaces=false, %
numbers=left, %
numberstyle=\tiny, %
breaklines=true, %
breakautoindent=true, %
captionpos=b,%
xrightmargin=0cm, %
xleftmargin=-0cm, %
language=tex, %
frameround=fttt;%
}

%%%%%%%%%%%%%%%% Lengths %%%%%%%%%%%%%%%%
\geometry{a4paper,twoside,left=2cm,right=2cm,marginparwidth=1.2cm,marginparsep=3mm,top=1.7cm,bottom=1.5cm}

\newcommand{\stamp}{{\tt \textit{Stamp }}}
\newcommand{\class}{{\tt \textit{class }}}
\newcommand{\initarg}{{\tt \textit{Initarg }}}
\bibliographystyle{plain}
\urlstyle{sf}

%%%%%%%%%%%%%%%%%%%%%%%%%%%%%%%%%%%%%%%%%%%%%%%

\newenvironment{vcenterpage}
{\newpage\vspace*{\fill}}
{\vspace*{\fill}\par\pagebreak}

\newtheorem{ex}{Exemple}%[section]
\newtheorem{theo}{Theroem}
\newcommand{\tuple}[1]{\ensuremath{\langle #1 \rangle}}

\begin{document}
%%%% Page de titre %%%%
% =============== 
% set of variables 
% ===============

% the title
\def\presentation{PER PROJECT DEFENCE}
\def\bottomTitle{PER PROJECT}
\def\noteAboutAuthor{Engeneering students/ENSEIRB-MATMECA}
\def\subject{Grid deformation for data visualization}
%\def\subject{Déformation de Grille pour la visualisation d'information}
% =================
% Header
% =================
\title[\bottomTitle]{
        {\bfseries \huge \presentation\\} 
        {\bfseries \subject}\\
        {\small\bf A. Lambert, R. Bourqui, D. Auber}\\   
}

\titlegraphic{
  \includegraphics[scale=0.15]{../rapport/img/logobordeaux1.jpg}
  \hspace{2cm}
  \includegraphics[scale=0.25]{../rapport/img/noms.png}
  \hspace{2cm}
  \includegraphics[scale=0.25]{../rapport/img/logo.jpg}
}

\author[\noteAboutAuthor]{
  {\normalsize \bfseries \sffamily Clients : }
  David {\sc Auber} \hspace{1cm} Romain {\sc Bourqui}\\    
}



%%%% remerciement %%%%%%%
\input{src/remerciement}
\addcontentsline{toc}{chapter}{Thanks}

%%%%% terminologie %%%%%%%%%%
\input{src/terminologie}

%%%% resume %%%%%%%%%
\begin{abstract}
\input{src/resume}
\end{abstract}

%%%% plan %%%%%
\tableofcontents
\listoffigures
%\listoftables



%%%%% corps du rapport %%%%%%%%%
\chapter*{Introduction}

The article summarized in this document was written by A.Lambert, R.Bourqui and D . Auber, researchers in LaBRI in Bordeaux.


This article speaks about how to visualize graphs containing many nodes and edges. With improvements in data acquisition comes an increase of the size and the complexity of graphs and this huge amount of data generally causes visual clutter, in our case due to edges crossing.
For example, it could be interesting to visualize data in fields like biology, social sciences, data mining or computer science and then emphasize their high-level pattern to help users perceive underlying models.


Nowadays, in the research world, the information is easily represented into graphs to visualize more and more data. However, this huge amount of information prevents the graph from being manually drawn:  It explains the need of software able to generate an appropriate graph with all nodes and edges. Yet this graph may suffer from cluttering, which should be reduced for a better understanding.


The first part of this document presents review related work on reducing edge clutters and enhancing edge bundles visualization, with which the article is connected. The second will deal with the Tutte algorithm , followed by the implementation issues. A third part will show our results. Finally, we draw a conclusion and explain the limits of our work for further improvements.

\addcontentsline{toc}{chapter}{Introduction}

\subsubsection{Distribution of nodes: Graph coloring}
In order to implement a parallel asynchronous version of Tutte algorithm, it is necessary to separate graph nodes into different sets. The objectif is to extract an independance between nodes. In fact, each node have to move while maintaining neighbors positions. Thus, the independance must be between node and his neighbors. This problem is similar to the famous problem of graph colorating.
\paragraph*{}
The objectif of the modified Tutte algorithm is to handle graph of thousands nodes. To separate such a number of nodes, it is more performant to use a heuristic of the algorithm of graph coloring.\\
The greedy algorithm is a simple and good solution to separate nodes into sets fastly and effictively.
\paragraph*{}
The algorithm used in this project is :
\begin{verbatim}
G={V,E}
Y = V
color = 0
While Y is not empty
   Z = Y
   While Z is not empty
      Choose a node v from Z
      Colorate v with color
      Y = Y - v
      Z = Z - v - {neighbors of v}
   End while
   color ++
End while
\end{verbatim}
\subsubsection{Applying Tutte algorithm to sets}
Once a sets of nodes is obteined, it is possible to apply a parallel Tutte algorithm. The question is how to parallelize it on sets of nodes. The natural idea is to attribute one sets per thread :
\begin{figure}[!h]
\centering
\includegraphics[scale=0.5]{img/distribution_verticale.png}
\caption{One set per thread}
\end{figure}
This distribution is far from being optimal. In fact, each thread have to lock neighbors node before moving it, wich introduce an importante critical section. In addition to being unfair, this distribution is limited by number of sets produced.\\

The best distribution for sets obteined by graph coloring is to execute $n$ threads on one set. Each thread move an number of nodes of the set without any critical section. This is because each node of the set is not the neighbor of all others nodes of the same set. Once the thread finish moving all his nodes of the set, he must wait for others threads (implemented by a barrier). After, the same processus is applyed on the next set.

\begin{figure}[!h]
\centering
\includegraphics[scale=0.5]{img/distrib.png}
\caption{$n$ threads per set}
\end{figure}


Reference : Introduction to Algorithms (Cormen, Leiserson, Rivest, and Stein) 2001, Chapter 16 "Greedy Algorithms".


\chapter{The context}

%\frame{
  \frametitle{The context}
\begin{center} 
Easy understanding with effective representation
 \end{center}
  \begin{figure}[H]
    \centering
    \includegraphics[scale=0.29]{../rapport/img/graphes_jolis.png}
  \end{figure}
  \pause
  \begin{alertblock}{}
    Only available for few graphs
\end{alertblock}
\vspace{1cm}
} 

\frame{
  \frametitle{The context}
  \framesubtitle{Two different methods}
  \begin{figure}[H]
    \centering
    \includegraphics[scale=0.4]{../rapport/img/slide3.jpg}
  \end{figure}
 
\vspace{1cm}
} 

\frame{
  \frametitle{The context}
  \framesubtitle{Compound visualization}
  \begin{figure}[H]
    \centering
    \includegraphics[scale=0.4]{../rapport/img/slide4.jpg}
  \end{figure}
 
\vspace{1cm}
} 

\frame{
  \frametitle{The context}
  \framesubtitle{Compound visualization}
  
  \begin{columns}[!ht]
    \begin{column}{4cm}
    \begin{block}{}   
     \begin{itemize}
     \item Nodes gathered into metanodes\\
     \item Inter-cluster edges merged into metaedges
     \end{itemize}
     \end{block}
    \end{column}
    \begin{column}{6cm}   
      \begin{figure}[H]
        \centering
        \includegraphics[scale=0.5]{../rapport/img/slide5.jpg}
      \end{figure}
    \end{column}
  \end{columns}
 \pause

  \begin{alertblock}{}
  \centering
    Impossibility for some nodes to move:
    node positions bring information
\end{alertblock}

\vspace{1cm}
} 

\frame{
  \frametitle{The context}
  \framesubtitle{Edge bundling}
  \begin{figure}[H]
    \centering
    \includegraphics[scale=0.4]{../rapport/img/slide6.jpg}
  \end{figure}
 
\vspace{1cm}
} 

\frame{
  \frametitle{The context}
  \framesubtitle{Edge bundling}

\begin{alertblock}{}
  \centering
    Impossibility for some nodes to move:
    node positions bring information
\end{alertblock}
\pause
\begin{figure}[H]
        \centering
        \includegraphics[scale=0.2]{../rapport/img/fleche.png}
      \end{figure}
\begin{center}
\begin{exampleblock}{}
\centering
Keep node position but edge aggregation
\end{exampleblock}
\end{center}                 
\pause 
 \begin{columns}[!ht]
\vspace{1cm}
    \begin{column}{6cm} 
    \begin{block}{}
    \centering
    Routes edges into bundles
    \end{block}
    \end{column}
    \begin{column}{6cm}   
      \begin{figure}[H]
        \centering
        \includegraphics[scale=0.5]{../rapport/img/slide7_2.jpg}
      \end{figure}
    \end{column}
  \end{columns}

} 


\chapter{Initialisation}

\section{Grid Presentation}
%it is asked here to talk about the situation of the graph is before we start the implementation: how do we manage to have the graph

\section{Tutte Algorithm}
%It is asked here to talk briefly about the algorithm we are going to use in the project: how it works (hope you see what I mean) 
\input{src/tutte}




\chapter{Implementations Issues}

\section{Naive implementation}

\subsection{Data Structure}
%here sould be presented the first data structure used
In the implementation of our solution we have defined our own data
structure on which we execute the Tutte algorithm. We have
implemented some mechanisms to convert a tulip format graph to our own
graph structure and also to get information from our structure
inserted in a Tulip format graph. In other words our structure is a
temporary structure for storing information about nodes in order to
execute the Tutte algorithm.

\subsection{Issues}
As a Tulip Data Structure contains a lot of information, it is expensive to
manipulate them. Furthermore, we do not need all the information
from a given Tulip graph. Especially, we do not need all
the properties about nodes to conduct the Tutte algorithm. For
instance, for a given node we just want to know if it is fixed, for a
fixed node, position never changes during the Tutte algorithm. In
addition to that, as we are looking for performance, we need a light structure matching the principe of Tutte algorithm. the following points are the criteria which we selected to set up a new data structure.
\begin{enumerate}
\item The fact that a given node is fixed or not is indicated firstly
  by a mobility property. However, there is another property indicating
  nodes which are part of graph contouring, and these nodes need to be
  fixed too. Therefore, to deal with the fact that a given node is fixed or
  not, we need to manipulate two properties that cost a lot.

\item In Tulip data structure there is a hierarchy of graphs. However, we only need
  the parent of the graph, the one which is not subgraph of another
  one. we do not need the sub-graph relation between graphs.

%% \item The informations about nodes are not in node but there is a
%% map between nodes and properties. For a given node, it cost a lot
%% to acces to one of its properties.

\end{enumerate}  

\subsection{First implementation}
In order to avoid memory leak and implement the Tutte
algorithm easily, we merely store only the information
needed to run the algorithm in our structure. We implemented our structure so that one
can easily access the neighbourhood of a given node for it is very
crucial in a Tutte algorithm implementation. To do this we define a
class that contains the various data needed on a given node (the
attibutes) and all the operations we need to run on a node (the
methods).

\newpage
\begin{lstlisting}
class MyNode {
 private:
  node n;
  bool mobile;
  Coord coord;  
  vector<MyNode *> voisin;

 public:
  MyNode();
  MyNode(const node n, const Coord coord);
  MyNode(const node n, const bool mob, const Coord coord);
  ~MyNode();
  
  const node getNode() const;
  bool getMobile() const;
  void setMobile(const bool b);
  const Coord getCoord() const;
  void setCoord(const Coord &);
  vector<MyNode *> * getVoisin();
  vector<MyNode *> getVoisin() const;
};
\end{lstlisting}

\subsubsection{The vertex attributes needed}
\begin{dinglist}{70}
\item[n]: this attribute is of \texttt{node} type of \textsf{Tulip} library and contains the \texttt{ID} of the node.  
\item[mobile]: this attribute is of \texttt{boolean} type and is used to know a given node is considered fixed.
\item[coord]: this attribut is of \texttt{Coord} type of \textsf{tulip} library and is used to store the node coordinates. 
\item[voisin]: this attribute is of \texttt{vector} type of \texttt{C++} library and contains the neighbourhood.
\end{dinglist}

\subsubsection{The operations on a vertex}
We used two types of operations or methods: \textsf{setter} and
\textsf{getter}. A \textsf{setter} is a method used to set the value
of an attribut and a \textsf{getter} is used to get the value of an
attribut. For a given attribut \texttt{attribut} , the corresponding
setter and getter are respectively \verb+setAttribut(args)+ and \verb+getAttribut()+. Below are the lists of the setters and getters of nodes in our structure:
\begin{dinglist}{70}
\item[Setters]: \verb+getMobile(), getCoord(), getVoisin()+  
\item[Getters]: \verb+setMobile(const bool b), setCoord(const Coord &), getVoisin()+.
\end{dinglist}

\subsection{Second implementation}
\subsection{Third implementation}
%\subsection{Enhanced implementation}
\newpage

\subsubsection{Distribution of nodes: Graph coloring}
In order to implement a parallel asynchronous version of Tutte algorithm, it is necessary to separate graph nodes into different sets. The objectif is to extract an independance between nodes. In fact, each node have to move while maintaining neighbors positions. Thus, the independance must be between node and his neighbors. This problem is similar to the famous problem of graph colorating.
\paragraph*{}
The objectif of the modified Tutte algorithm is to handle graph of thousands nodes. To separate such a number of nodes, it is more performant to use a heuristic of the algorithm of graph coloring.\\
The greedy algorithm is a simple and good solution to separate nodes into sets fastly and effictively.
\paragraph*{}
The algorithm used in this project is :
\begin{verbatim}
G={V,E}
Y = V
color = 0
While Y is not empty
   Z = Y
   While Z is not empty
      Choose a node v from Z
      Colorate v with color
      Y = Y - v
      Z = Z - v - {neighbors of v}
   End while
   color ++
End while
\end{verbatim}
\subsubsection{Applying Tutte algorithm to sets}
Once a sets of nodes is obteined, it is possible to apply a parallel Tutte algorithm. The question is how to parallelize it on sets of nodes. The natural idea is to attribute one sets per thread :
\begin{figure}[!h]
\centering
\includegraphics[scale=0.5]{img/distribution_verticale.png}
\caption{One set per thread}
\end{figure}
This distribution is far from being optimal. In fact, each thread have to lock neighbors node before moving it, wich introduce an importante critical section. In addition to being unfair, this distribution is limited by number of sets produced.\\

The best distribution for sets obteined by graph coloring is to execute $n$ threads on one set. Each thread move an number of nodes of the set without any critical section. This is because each node of the set is not the neighbor of all others nodes of the same set. Once the thread finish moving all his nodes of the set, he must wait for others threads (implemented by a barrier). After, the same processus is applyed on the next set.

\begin{figure}[!h]
\centering
\includegraphics[scale=0.5]{img/distrib.png}
\caption{$n$ threads per set}
\end{figure}


Reference : Introduction to Algorithms (Cormen, Leiserson, Rivest, and Stein) 2001, Chapter 16 "Greedy Algorithms".


\subsection{Tutte Algorithm}
%here should be the implementation of Tutte with basic structure 

\section{First Optimization}
%in this first optimization Frozar you are suggested to put the improvement due to the new data structure you took

\section{Second Optimization}
%in this optimization Frozar you are suggested to talk about results with openMP

\section{third Optimization}
%in this optimization Ozenati you are suggested to talk about the asynchronous implementation (if faster than the others) with graph coloration

\chapter{Level of reduction}

\section{Improvements of our implementation}
%somebody is suggested to talk about limits of Tutte algorithm on some cases of grid and why our implementation is better than what we had at the beginning
\section{Results on graphs}
%compare Benchmarks on the graphs given and the different implementations and improvements 


\addcontentsline{toc}{chapter}{Conclusion}
\frame{
  \frametitle{Thank you for your attention!}
  \begin{exampleblock}{}
  \begin{center}
    {\huge Any question?}
  \end{center}
  \end{exampleblock}
}

%I suggest here to talk about what can be do in the future (there is a quite start of this in my context written in Wiki) 
\begin{thebibliography}{}

\bibitem{pa} E. Colin de Verdière, M. Pocchiola, and G. Vegter. Tutte's Barycenter Method applied to Isotopies. \emph{Computational Geometry: Theory and Applications, 26}, 81–97, 2003.

\bibitem{pb} B. Bollob's. Modern graph theory, \emph{volume 184 of Graduate Texts in Mathematics}. Springer-Verlag, 1998.

\bibitem{pc} W. T. Tutte. How to draw a graph. \emph{Proceedings of the London Mathematical Society}, 13:743–768, 1963.

% \bibitem{1} 

% \bibitem{2} 

% \bibitem{3} 

% \bibitem{4} 

% \bibitem{5} 
   
\end{thebibliography}


%%%%%% Liste des annexes %%%%%%%%%%
\part{Annexes}
\appendix
\input{src/annexeA}
\end{document}
