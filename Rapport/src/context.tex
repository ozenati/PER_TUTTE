\textbf{Some classes of graph}, such as trees or acyclic graphs, clearly facilitate user understanding by effective representation. However, most of graphs do not belong to these classes and algorithm giving nice results in terms of time and space complexity but also in terms of aesthetic criteria for any graph do not exist yet. For example, force-directed method produces pleasant and structurally significant results but do not help users comprehension due to data complexity. Authors of the paper specify two techniques for that reduction: the compound visualization and the edge bundling. But their interest goes to Edge Bundling which suggest to route edges into bundles in order to uncover high-level edge patterns and emphasize information. Their contribution was to set this edge bundling by discrediting the plane into a new specific domain where the graph is set so that boundaries of the new discretization.
\\
\\
\textbf{Up to now}, several techniques were used to reduce this clutter, based on compound visualization or edge bundling. In a compound visualization, nodes are gathered into metanodes and inter-cluster edges are merged into metaedges. To retrieve the information, metanodes could be collapsing or expanding. Edge bundling technique routes edges into bundles. This uncover high level edge pattern and emphasize relationships.
Yet an important constraint is the impossibility for some nodes to move while avoiding edges crossing because node positions bring information: the compound visualization is consequently not suitable.
\\
\\
\textbf{Some existing representations} take into account this duty: some reducing edge clutters (Edge routing, Interactive techniques, Confluent Drawing, Node clustering, Edge clustering) and the other enhancing edge bundles visualization (Smoothing curves, Coloring edges). Edge routing use shortest-path edge routing to bound the number of edge crossings and use non-point-size to avoid node-edge overlaps. It do not highlight the underlying model. Interactive techniques remove clutter around the user’s focii in a fisheye-like manner while preserving node position: this technique does not reduce the clutter of the entire representation. In Confluent Drawing, groups of crossing edges are drawn as curved overlapping lines. Node Clustering routed edge along the hierarchy tree branches. Both methods cannot be applied to any graph.
Edge Clustering route edges either on the outer face of the circle or in its inner faces and bundle them to optimize area utilization.
\\
\\
\textbf{The publication we are currently working on} is based on a bundling edges method using quad-tree discretization, Voronoï diagrams and Bezier curves to visualize aesthetic graphs. By using the shortest path Dijskstra  algorithm and optimizing the algorithm using efficiently the architecture of computer, the obtained graphs are also quickly computable. The Bezier curves are quite interesting because they reduce the “zigzag” effect which appears in edge weighting, However, the graph produced by this Bezier curves are very poor in information: because of edges superposition over nodes, information is lost when edge weights are adjusted. Moreover, the graph is not as “aesthetical” as it should be.
