In the implementation of our sulution we have defined our own data
structure on which we excute the Tutte's algorithm. We have
implemented some mechanisms to convert a tulip format graph to our own
graph structure and also to get informations from our structure
inserted in a tulip format graph. In other words our structure is a
temporary structure for storing informations about nodes in order to
execute the Tutte's algorithm.

\subsection{Aim}
As a format tulip contains a lot of informations so it costs a lot to
manipulate them and so far away we do not need all the informations
about a given tulip graph. To point out we especially do not need all
the properties about nodes to conduct the Tutte's algorithm.For
intance, for a given node we just want to know if it is fixed for a
fixed node's position nevers changes during the tutte's algorithme. In
addition of that, as we are looking for performance, we need a light
and adapted structure to the principe of Tutte's algorithm. Below are
some of criterias which make us thought we need a new data structure.
\begin{enumerate}
\item The fact that a given node is fixed or not is indicated firsly
  by a mobility property. However there is anover property incating
  nodes wich are part of graph contouring and these nodes need to be
  fixed too. So to deal with the fact that a given node is fixed or
  not we need to manipulate two properties that cost a lot.

\item In tulip format there is an hierarchy of graphs or we only need
  the parent of the graph, the one which is not subgraph of another
  one. we do not need the sub-graph relation between graphs.

%% \item The informations about nodes are not in node but there is a
%% map between nodes and properties. For a given node, it cost a lot
%% to acces to one of its properties.

\end{enumerate}  

\subsection{First implementation}
In order to avoid memory leak and implemente easily the Tutte's
algorithme, we merely stock in our sttructure only the informations
needed to run the algorithm. We implemented our structure so that one
can easily access the neighbourhood of a given node for it is very
crucial in am Tutte'algorithm implementation. To do this we define a
class that contains diferent informations needed on a given node (the
attibuts) and all the operations we need to run on a node (the
methods).

\newpage
\begin{lstlisting}
class MyNode {
 private:
  node n;
  bool mobile;
  Coord coord;  
  vector<MyNode *> voisin;

 public:
  MyNode();
  MyNode(const node n, const Coord coord);
  MyNode(const node n, const bool mob, const Coord coord);
  ~MyNode();
  
  const node getNode() const;
  bool getMobile() const;
  void setMobile(const bool b);
  const Coord getCoord() const;
  void setCoord(const Coord &);
  vector<MyNode *> * getVoisin();
  vector<MyNode *> getVoisin() const;
};
\end{lstlisting}

\subsubsection{The vertex's attributs needed}
\begin{dinglist}{70}
\item[n]: this attribut has \texttt{node} type of \textsf{tulip} library and contains the \texttt{ID} of the node.  
\item[mobile]: this attribut has a \texttt{boolean} type and used to know a given node is considered fixed.
\item[coord]: this attribut has a \texttt{Coord} type of \textsf{tulip} library and is used to stocked the node coordinate. 
\item[voisin]: this attribut has a \texttt{vector} type of \texttt{C++} library and contain the neighbourhood.
\end{dinglist}

\subsubsection{The operations on a vertex}
We used two types of operations or methods: \textsf{setter} and
\textsf{getter}. A \textsf{setter} is a method used to set the value
of an attribut and a \textsf{getter} is used to get the value of an
attribut. For a given attribut \texttt{attribut} , the cooresponding
setter and setter are respectively \verb+setAttribut(args)+ and \verb+getAttribut()+. Below are the lists of the setters and getters of nodes in our structure:
\begin{dinglist}{70}
\item[Setters]: \verb+getMobile(), getCoord(), getVoisin()+  
\item[Getters]: \verb+setMobile(const bool b), setCoord(const Coord &), getVoisin()+.
\end{dinglist}

%\subsection{Enhanced implementation}
\newpage
