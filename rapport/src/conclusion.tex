\chapter*{Conclusion}


With our best Tutte algorithm implementation , we obtain an algorithm convergence in 5 iterations and it spends x seconds to recalculate all the coordinates. \textbf{(à mettre à jour)}
Finally, our optimizations allowed an improvement of the clutter reduction and the performance compared with existent methods.


After discussing with our teachers in charge, some issues concerning our standalone version remains unresolved and could be taken into account in the future: Rather than working with input graph nodes, it would be interesting to dynamically add nodes (and their edges to keep a triangular-face graph) and launch again the algorithm in order to refine given results. It will also be quite interesting to automatically join small triangles or divide big triangles into littles ones.

Another interessant improvement would be to turn our standalone program to a Tulip plugin in order to integrate it in this software.
 
A theoritical point remains unsolved : despite the fact that the contour on which we apply the Tutte algorithm is not necessarily convex, we obtain correct results. It would be useful to understand why Tutte works on such kind of graphs: It could be caused by how the grid is built or by the following constraint: two fixed nodes can not be linked by an edge.


