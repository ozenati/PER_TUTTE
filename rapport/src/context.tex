\textbf{Some classes of graph}, such as trees or acyclic graphs, clearly facilitate user understanding by effective representation. However, most graphs do not belong to these classes, and algorithms giving nice results in terms of time and space complexity but also in terms of aesthetic criteria for any graph do not exist yet. For example, the force-directed method produces pleasant and structurally significant results but does not help user comprehension due to data complexity. The authors of the paper specify two techniques for that reduction: compound visualization and edge bundling. But their interest goes to Edge Bundling, which suggests to route edges into bundles in order to uncover high-level edge patterns and emphasize information. Their contribution was to set this edge bundling by discretizing the plane into a new mesh.
%domain where the graph is set so that boundaries of the new discretization are formed.
\\
\\
\textbf{Up to now}, several techniques have been used to reduce this clutter, based on compound visualization or edge bundling. In a compound visualization, nodes are gathered into metanodes and inter-cluster edges are merged into metaedges. To retrieve the information, metanodes could be collapsed or expanded. Yet an important constraint is the impossibility for some nodes to move while avoiding edges crossing because node positions provide information: consequently, compound visualization is not suitable. To reduce visual clutter, another clue is to keep vertices while edges are aggregated: the Edge bundling technique routes edges into bundles. This uncovers high level edge patterns and emphasizes relationships. Moreover, some existing representations take into account the the inability of some nodes to change position i.e.  some reducing edge clutters (Edge routing, Interactive techniques, Confluent Drawing, Edge clustering) and other enhancing edge bundles visualizations (Smoothing curves, Coloring edges). 
\\
\\
\textbf{The publication we are currently working on} is based on a new edge bundling algorithm for efficient graph drawing. By using specific  discretization methods such as quad-tree and Voronoï diagrams which are little time-consuming calculations, the authors obtained a new separation of the region where they can reduce the drawing area. As a result, their final discretization algorithm is a good trade off between good precision($quad~tree$) and the computation time ($Voronoï$)[$ref$].

% and Bezier curves to visualize aesthetic graphs.

In order to create the Edge-bundling effects, the authors use the "shortest path" Dijskstra  algorithm. But this does not create a decent number of bundles. Then they add new concepts such as $roads$ and $Highways$ to increase this number. This means to reduce the weight of an edge (of the grid obtained before) if it is highly used, but only after computing several shortest paths between linked nodes of the original graph. 

  Several optimizations of the specific shorstest path algorithm such as function calls reductions, multithreading reduces tremendously computation time of graphs.


The authors of the article, after the rendering of their algorithm, think that their grid can be improved because what they have got presents some inconveniences such as the big amount of bends ("zigzag" effect on the grid) and irregular triangles. To solve those problems, they want a heuristic to uniformize sizes of the edges and increase angles between edges that harm graph reading. The proposition in our work is the use of  the Tutte algorithm which is only applicable for an internally triangulated planar graph and renders more informative and aesthetic graphs (Less edge crossing, less bends, smaller sizes of edges, maximum angular resolution).
