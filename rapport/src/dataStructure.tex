In the implementation of our solution we have defined our own data
structure on which we execute the Tutte algorithm. We have
implemented some mechanisms to convert a tulip format graph to our own
graph structure and also to get information from our structure
inserted in a Tulip format graph. In other words our structure is a
temporary structure for storing information about nodes in order to
execute the Tutte algorithm.

\subsection{Issues}
As a Tulip Data Structure contains a lot of information, it is
expensive to manipulate them. Furthermore, we do not need all the
information from a given Tulip graph. Especially, we do not need all
the properties about nodes to conduct the Tutte algorithm. For
instance, for a given node we just want to know if it is fixed, for a
fixed node, position never changes during the Tutte algorithm. In
addition to that, as we are looking for performance, we need a light
structure matching the principe of Tutte algorithm. The following
points are the main reasons which lead us to set up a new data
structure.
\begin{enumerate}
\item The fact that a given node is fixed or not is indicated firstly
  by a mobility property. However, there is another property indicating
  nodes which are part of graph contouring, and these nodes need to be
  fixed too. Therefore, to deal with the fact that a given node is fixed or
  not, we need to manipulate two properties that cost a lot.

\item In Tulip data structure there is a hierarchy of graphs. However, we only need
  the parent of the graph, the one which is not subgraph of another
  one. we do not need the sub-graph relation between graphs.

%% \item The informations about nodes are not in node but there is a
%% map between nodes and properties. For a given node, it cost a lot
%% to acces to one of its properties.

\end{enumerate}  

\subsection{Implementations}
We tested three implementations in order to find out the right one. Because we care of memory leak, we merely store only the information needed to run the algorithm in our structure.

\subsubsection{First implementation}
In this implementation our structure is such that a given node
contains its neighbourhood. So one can easily access the neighbourhood
of a given node for it is very crucial in a Tutte algorithm
implementation. To do this we define a class that contains the various
data needed on a given node (the attibutes) and all the operations we
need to run on a node (the methods).

\newpage
\begin{lstlisting}
class MyNode {
 private:
  node n;
  bool mobile;
  Coord coord;  
  vector<MyNode *> voisin;

 public:
  MyNode();
  MyNode(const node n, const Coord coord);
  MyNode(const node n, const bool mob, const Coord coord);
  ~MyNode();
  
  const node getNode() const;
  bool getMobile() const;
  void setMobile(const bool b);
  const Coord getCoord() const;
  void setCoord(const Coord &);
  vector<MyNode *> * getVoisin();
  vector<MyNode *> getVoisin() const;
};
\end{lstlisting}

\paragraph{The vertex attributes needed}
\begin{dinglist}{70}
\item[n]: this attribute is of \texttt{node} type of \textsf{Tulip} library and contains the \texttt{ID} of the node.  
\item[mobile]: this attribute is of \texttt{boolean} type and is used to know a given node is considered fixed.
\item[coord]: this attribut is of \texttt{Coord} type of \textsf{tulip} library and is used to store the node coordinates. 
\item[voisin]: this attribute is of \texttt{vector} type of \texttt{C++} library and contains the neighbourhood.
\end{dinglist}

\paragraph{The operations on a vertex}
We used two types of operations or methods: \textsf{setter} and
\textsf{getter}. A \textsf{setter} is a method used to set the value
of an attribut and a \textsf{getter} is used to get the value of an
attribut. For a given attribut \texttt{attribut} , the corresponding
setter and getter are respectively \verb+setAttribut(args)+ and \verb+getAttribut()+. Below are the lists of the setters and getters of nodes in our structure:
\begin{dinglist}{70}
\item[Setters]: \verb+getMobile(), getCoord(), getVoisin()+  
\item[Getters]: \verb+setMobile(const bool b), setCoord(const Coord &), getVoisin()+.
\end{dinglist}

\subsubsection{Second implementation}

\subsubsection{Third implementation}
In this third implementation we do not use a class to store the
different informations about node to run Tutte algorithm. 
As we are looking for a more light data structure in order to ameliorate 
memory access, we use a struct to group datas needed about a given node
under one name 


%\subsection{Enhanced implementation}
\newpage
