\chapter*{Introduction}

The article summarized in this document was written by A.Lambert, R.Bourqui and D . Auber, researchers in LaBRI in Bordeaux.


This article speaks about how to visualize graphs containing many nodes and edges. With improvements in data acquisition comes an increase of the size and the complexity of graphs and this huge amount of data generally causes visual clutter, in our case due to edges crossing.
For example, it could be interesting to visualize data in fields like biology, social sciences, data mining or computer science and then emphasize their high-level pattern to help users perceive underlying models.


Nowadays, in the research world, the information is easily represented into graphs to visualize more and more data. However, this huge amount of information prevents the graph from being manually drawn:  It explains the need of software able to generate an appropriate graph with all nodes and edges. Yet this graph may suffer from cluttering, which should be reduced for a better understanding.


The first part of this document presents review related work on reducing edge clutters and enhancing edge bundles visualization, with which the article is connected. The second will deal with the Tutte algorithm , followed by the implementation issues. A third part will show our results. Finally, we draw a conclusion and explain the limits of our work for further improvements.
