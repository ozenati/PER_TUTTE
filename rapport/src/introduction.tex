\chapter*{Introduction}

The article talks about how to visualize graphs containing many nodes and edges. Improvements in data acquisition leads to an increase of the size and the complexity of graphs and this huge amount of data generally causes visual clutter, in our case due to edge crossing.
For example, it could be interesting to visualize data in fields like biology, social sciences, data mining or computer science, and then emphasize their high-level pattern to help users perceive underlying models.


Nowadays, in the research world, the information is easily represented into graphs to visualize more and more data. However, this huge amount of information prevents the graph from being manually drawn:  It explains the need of automatic methods able to generate an appropriate graph with all nodes and edges. Yet this graph may suffer from cluttering, which should be reduced for a better understanding.

Our objective all along this project is to read what has been done before relating to this problem, to provide an objective point of view on those previous works, and propose our contribution. We have implemented a method, then optimized its performances with
 current technologies (OpenMP, Tulip...) and setted our boundaries. 


The first part of this document presents review-related work on reducing edge clutters and enhancing edge bundle visualization, with which the article is connected. The second will deal with the Tutte algorithm and its differents versions. A third part will talk about the implementation issues and show our results. Finally, we draw a conclusion and explain the limits of our work for further improvements.
